\documentclass[11pt]{article}

\usepackage[margin=1in]{geometry}
\usepackage{enumitem}
\usepackage{hyperref}
\usepackage{titlesec}
\usepackage{graphicx}
\usepackage{array}
\usepackage[T1]{fontenc}
\usepackage[utf8]{inputenc}

\setlist{nosep,leftmargin=1.1em}
\titlespacing*{\section}{0pt}{0.6em}{0.3em}
\titlespacing*{\subsection}{0pt}{0.5em}{0.25em}
\setlength{\parskip}{0.3em}
\setlength{\parindent}{0pt}

\begin{document}

{\LARGE \textbf{GuideMe: AI-Powered Study Companion}}\\[-0.2em]
\textit{Personalized guidance, interactive diagrams, flashcards, and RAG-backed learning.}\\[0.6em]

\section*{Purpose \& Value}
GuideMe helps learners understand complex topics faster through conversational AI, on-demand visualizations (diagrams), and spaced-repetition flashcards. It blends chat-based assistance with retrieval-augmented generation (RAG) to ground answers in relevant sources, improving accuracy and trust. The goal is to turn passive reading into active, guided learning—summarize, visualize, quiz, and retain.

\section*{Core Features}
\begin{itemize}
  \item \textbf{Conversational Tutor}: Natural language chat that explains, summarizes, and explores topics step-by-step.
  \item \textbf{Diagram Generation}: Automatically creates Graphviz-based diagrams from descriptions to visualize systems \& flows.
  \item \textbf{Flashcards \& Quizzes}: Generate flashcards and quick quizzes from content to reinforce knowledge.
  \item \textbf{RAG (Retrieval-Augmented Generation)}: Pulls from user-provided or referenced documents to ground responses.
  \item \textbf{YouTube Ingestion}: Extracts and summarizes educational videos into study-ready notes and memory aids.
  \item \textbf{Study Hub}: Centralized workspace with chat history, flashcards, quizzes, and curated resources.
  \item \textbf{Inline Code/Diagrams}: Rich messages with code blocks, images, and structured content.
\end{itemize}

\section*{System Architecture \& Technologies}
\begin{tabular}{p{0.28\linewidth} p{0.68\linewidth}}
\textbf{Frontend/UI} & Django templates, Vanilla JS (dynamic chat, streaming), CSS for theming \\
\textbf{Backend} & Django (Python), Django REST where needed; streaming responses via HTTP chunking \\
\textbf{LLM Orchestration} & Tool-based agent system for choosing actions (e.g., diagram generator, flashcards) \\
\textbf{AI Models} & Groq API for chat completions; Google Gemini (e.g., \texttt{gemini-2.5-flash}) for structured tasks/vision \\
\textbf{RAG} & Document chunking \& retrieval; vector indexing for semantic lookups \\
\textbf{Storage/DB} & PostgreSQL (Neon in production); media storage for generated diagrams \\
\textbf{Diagrams} & Graphviz (via \texttt{graphviz} Python package; \texttt{dot} binaries) \\
\textbf{Auth} & Django auth + Allauth (Google sign-in support) \\
\textbf{Security} & CSRF, session hardening, HTTPS, static file integrity via WhiteNoise \\
\textbf{Static Assets} & WhiteNoise (compressed static files), collectstatic in entrypoint \\
\end{tabular}

\section*{Deployment \& DevOps}
\begin{itemize}
  \item \textbf{Containerization}: Docker (Python slim base, system deps for Graphviz).
  \item \textbf{Platform}: Railway (serverless mode enabled, builds from repo).
  \item \textbf{CI/CD}: GitHub Actions (lint, test, security scan, image build); Railway auto-deploys from main.
  \item \textbf{Runtime}: Gunicorn (workers/threads tuned), health checks, entrypoint runs migrations \& collectstatic.
\end{itemize}

\section*{Security \& Privacy}
\begin{itemize}
  \item Environment-based secrets, least-privilege tokens, dependency scanning.
  \item Session security (secure cookies, same-site), CSRF protection.
  \item RAG respects document boundaries; no data persisted beyond intended scope.
\end{itemize}

\section*{Typical Workflow}
\begin{enumerate}
  \item User asks a question or provides content (PDF/YouTube).
  \item Agent chooses tools: retrieve context (RAG), generate explanation, optionally create diagram.
  \item System returns a structured response with text, code/diagram, and suggested flashcards.
  \item User saves flashcards/quizzes and revisits via Study Hub.
\end{enumerate}

\section*{Why GuideMe}
\textbf{Guided understanding} (diagrams), \textbf{grounded answers} (RAG), and \textbf{memory reinforcement} (flashcards)---in one streamlined study companion.

\vfill
\footnotesize{
Tech stack: Django, PostgreSQL (Neon), Graphviz, Groq LLMs, Google Gemini (\texttt{gemini-2.5-flash}), WhiteNoise, Docker, Railway, GitHub Actions.
}

\end{document}